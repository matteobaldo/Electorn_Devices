%------------------------------------------------------------------------%
\chapter{BJT}
%------------------------------------------------------------------------%

\centering
{\bf Figure 102}\\
\raggedright

This device is created with 2 pn junctions with a back to back connection creating a npn or a pnp device. This device can be horizontal or vertical (we will study this second approach beacuse is the most important).\\
The n structure is the collector the $n^+$ is the sub-collector region connected to the surface with a reach-throught implantation. The current flows in the vertical direction so the most significant part is the one under the collector area.\\
The main idea is to forward bias the EB-j and reverse bias the BC-j. Foward bias in EB-j so there will be a flow of electrons form E to B holes to B-E but if the base region is narrow we have that electrons don't recombine in the p region but travel throught it reaching the collector.\\
We have a large flow of electrons throght a reverse biased junction (BC) beacuse the supply comes from the EB-j and not from GR processes. In the collector we have high resistivity due to low doping concentrations so we have to put the sub-collector to make a contact.\\
\vspace{5mm}
This device has a significant flow of both electrons and holes so it's a bipolar device (intrinsic flow of holes from base to emitter).\\
We can minimize the base current by reducing the width of the base and olso making the emitter more doped wrt the base.\\
\vspace{5mm}
If the BC-j was direct biased we will have an additional component to the base current that is something we want to avoid.\\

\section{Doping concentrations}
\centering
{\bf Figure 103}\\
\raggedright
Let's analyse the doping concentrations; in emitter region we have high doing concentration $N_d^E\simeq 10^{20}$ that decrease until $10^{18}$, the base from $N_a^B=10^{18}$ to $10^{17}$ and then the collector and the subcollector as in figure. The non constant doping concentration is relevant in the operation of the device.\\
 
\section{Collector current}
We can focus our analysis on the quasi-neutral p region that is the bottom-neck for conduction of the system.

\centering
{\bf Figure 107down}\\
\raggedright

We call $W_b$ the quasi neutral region of the base. In this region the band banding is set by holes concentration. Doping concentration is not constant so we have a band banding like in figure so a F due to non constant doping concentrations. This condition can be reached olso with high level of injection.\\
Considering hole concentration $p=n_ie^{\frac{E_i-E_{fp}}{kT}}$ we get $E_i=E_{fp}+kT\ln(p/n_i)$. We want this parameter beacuse it's releted to the potential and so to the electric field
\begin{equation}
F=-\frac{d\phi}{dx}=-\frac{1}{q}\frac{dE_{i}}{dx}=\frac{1}{q}\frac{dE_{fp}}{dx}+\frac{kT}{q}\frac{1}{p}\frac{dp}{dx}
\end{equation}
both high injection and non constant doping concentration events are taken into account by the term $\frac{dp}{dx}$, the term $\frac{dE_{fp}}{dx}$ take into account of resistive drops.\\
\vspace{5mm}
The base current density is $J_B=p\mu_p \frac{dE_{fp}}{dx}$ so we get that $\frac{dE_{fp}}{dx}=J_p/(p\mu_p)$ and so putting this equation into the electric field equation we get 
\begin{equation}
\frac{1}{q}\frac{J_p}{p\mu_p}+\frac{kT}{q}\frac{1}{p}\frac{dp}{dx}
\end{equation}
using some reasonable values for all the parameter we get that $\Delta V_t$ is of a few fractions of mV when the total voltage applied is in the order of 0.6-0.7V. This contribution is negligible we consider the $E_{fp}$ almost constant so 
\begin{equation}
F=\frac{kT}{q}\frac{1}{p}\frac{dp}{dx}
\end{equation}

\vspace{5mm}
We can recover the drift-diffusion expression for the current and using F we get
\begin{equation}
J_n=qn\mu_n \frac{kT}{q}\frac{1}{p}\frac{dp}{dx}+qD_n \frac{dn}{dx}
\end{equation}
We are in a quasi neutral region so $p(x)=N_a(x)+n(x)$ we can get 2 cases.\\
\vspace{3mm}
\tab {\bf Low injection}\\
$n<<N_a$ so the expression for the current is 
\begin{equation}
J_n=qn\mu_n \frac{kT}{q}\frac{1}{N_a}\frac{dN_a}{dx}+qD_n \frac{dn}{dx}
\end{equation}
where $\frac{kT}{q}\frac{1}{N_a}\frac{dN_a}{dx}$ is the built in electric field due to non constant doping concentration.\\
\vspace{3mm}
\tab {\bf High injection}\\
$n>>N_a$ so we get 
\begin{equation}
J_n=qn\mu_n \frac{kT}{q}\frac{1}{n}\frac{dn}{dx}+qD_n \frac{dn}{dx}=2qD_n\frac{dn}{dx}
\end{equation}
it's like a pure diffusion process with the double of the intesity; this is called the Webster effect.\\

\subsection{Prototype BJT current}
\centering
{\bf Figure 105up}\\
\raggedright
It's a super-ideal case where we have constant doping concentrations and low level of injection.\\
So we have that $J_n=qD_n \frac{dn}{dx}$; we can say as "boundary conditions" that $n(0)=\frac{n_i^2}{N_a}e^{qV_{BE}/kT}$ and that $n(W_b)=\frac{n_i^2}{N_a}e^{qV_{BC}/kT}$ but beacuse $V_{BC}<0$ we can say that $n(W_b)\simeq 0$.\\
Now we can integrate the two members and since $J_n$ is a constant becuse there isn't recombination we get
\begin{equation}
J_n=qD_n\frac{n(W_b)-n(0)}{W_b}
\end{equation}
that is legit beacuse $\frac{dn}{dx}$ it's a linear profile we are considering a narrow base diod.\\
\begin{equation}
J_n=-\frac{qD_nn_i^2}{N_aW_b}e^{qV_{BE}/kT}
\end{equation}
We can now multiply for the area of the emitter $A_E$ and add a minus sign beacuse we want current positive if flows into the collector arriving at
\begin{equation}
I_c=A_E\frac{qn_i^2}{\frac{N_aW_b}{D_n}}e^{qV_{BE}/kT}=A_E\frac{qn_i^2}{G_B}e^{qV_{BE}/kT}
\end{equation}
where $G_B=\frac{N_aW_b}{D_n}$ it's the Gummel number of the base.\\
This current it's dependent only on base parameters.

\subsection{"Ideal" BJT}
We consider now the possibility of non constant doping concentration or high injection regime so we recover the expression $J_n=qn\mu_n \frac{kT}{q}\frac{1}{p}\frac{dp}{dx}+qD_n \frac{dn}{dx}$ that we can write as
\begin{equation}
J_n=\frac{qD_n}{p}\left(n \frac{dp}{dx}+p \frac{dn}{dx}\right)= \frac{qD_n}{p}\frac{d(pn)}{dx}
\end{equation}
As before we integrate keeping $J_n$ as a constant we know that $pn(W_d)\simeq 0$ so we reach the current density as 
\begin{equation}
J_n=-\frac{qn_i^2}{\int^{W_b}_0 \frac{p}{D_n}dx}e^{qV_{BE}/kT}
\end{equation}
If we multiply for the emitter area and change sign we get the collector current as 
\begin{equation}
I_c=A_E\frac{qn_i^2 e^{qV_{BE}/kT}}{\int^{W_b}_0 \frac{p}{D_n}dx}=A_E\frac{qn_i^2 e^{qV_{BE}/kT}}{G_B}
\end{equation}
where $G_B=\int^{W_b}_0 \frac{p}{D_n}dx$ is the new Gummel number for the base that in case of constant doping and low level of injection becomes the first one.\\

\subsection{Real BJT}
The most general case we can include the non-constant bandgap in the base. We can have a non constant bandgap due to 2 causes.\\
The first cause of bandgap narrowing is due to doping conectrations that are higher than $\simeq 10^{18}$; high doping means donor closer to each other interaction between energy levels that becomes a band of energy and some of this level can overlap the $E_c$ narrowing the band-gap. The emitter side we will have a narrower gap than the collector side.\\
Second cause is a technological method to gradually change from Si to GeSi to narrow the bandgap ($E_{gap}^{Ge}<E_{gap}^{Si}$).
In all the calculation we have an important role of the bangap only in $pn=n_i^2e^{(E_{fn}-E_{fp})/kT}$ in this the intrinsic carrier concentration becomes
\begin{equation}
n_{i,e}^2=N_vN_ce^{-E_{gap}/kT}e^{\Delta E_{gap}/kT}=n_i^2e^{\Delta E_{gap}/kT}
\end{equation}










































